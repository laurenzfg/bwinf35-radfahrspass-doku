\documentclass[parskip=full, DIV=14]{scrartcl}

\usepackage[T1]{fontenc}
\usepackage{selinput}% Eingabecodierung automatisch ermitteln siehe <http://ctan.org/pkg/selinput>
\SelectInputMappings{
	adieresis={ä},
	germandbls={ß},
} 
\usepackage{ngerman}

\usepackage{amsmath}

\usepackage{graphicx}
\graphicspath{{Bilder/}}

\usepackage{listings}
\usepackage[dvipsnames]{xcolor}

\lstset{language={C++},
        numbers=left,
        stepnumber=1,
        numbersep=5pt,
        numberstyle=\tiny,
        breaklines=true,
        breakautoindent=true,
        postbreak=\space,
        tabsize=2,
        basicstyle=\ttfamily\footnotesize,
        showspaces=false,
        showstringspaces=false,
        extendedchars=true,
        commentstyle=\color{Gray}, % comment color
    	keywordstyle=\color{Bittersweet}, % keyword color
    	stringstyle=\color{Orange}, % string color
        % Sorgt dafür, dass das Paket listings auch mit den Sonderzeichen in UTF-8 zurecht kommt.
				literate=
					{Ö}{{\"O}}1
					{Ä}{{\"A}}1
					{Ü}{{\"U}}1
					{ß}{{\ss}}2
					{ü}{{\"u}}1
					{ä}{{\"a}}1
					{ö}{{\"o}}1
        }

\usepackage{scrpage2}
\usepackage[hidelinks]{hyperref}

\newcommand{\shellcmd}[1]{\texttt{\$ #1}\\}
\newcommand{\shellout}[1]{\texttt{#1}\\}

\begin{document}
	\pagestyle{scrheadings}

	\ihead{Aufgabe 4: Radfahrspaß, Team 00001}\ohead{Jon Fehling, Laurenz Grote}

	\titlehead{Aufgabe 4: Radfahrspaß, Team 00001 \hfill Jon Fehling, Laurenz Grote}
	\title{Radfahrspaß}
	\subtitle{Aufgabe 4}
	\author{Jon Amos Fehling \& Laurenz Friedrich Grote}
	\date{}
	\maketitle
	\tableofcontents
	
	\vspace {2em}
	Unsere Umsetzung für "`Radfahrspaß"' erfolgte unter Ubuntu mit C++/C (Compiler: g++ 5.4.0). Das Programm liegt als Quellcode vor.
	\clearpage
	% ----------------------------------------------------------------------------
	\section{Lösungsidee}
		\subsection {Lösungsidee}
	Die Strecke wird mit einer theoretisch minimalen Geschwindigkeit und einer theoretisch maximalen Geschwindigkeit befahren. 
	Für die maximale Geschwindigkeit wird auf jedem geraden Streckenabschnitt beschleunigt, für die minimale Geschwindigkeit zunächst auf jedem geraden Streckenabschnitt gebremst.

	Unterschreitet die minimale Geschwindigkeit jedoch 0, wird diese um 2 erhöht, sodass dieser Punkt mit einer positiven Geschwindigkeit erreicht wird.
	Diese Erhöhung entspricht der nachträglichen Umwandlung eines vergangenen Abbremsvorganges in einen Beschleunigungsvorgang. Die Geschwindigkeit wird um zwei erhöht, da ein Bremsvorgang (-1) in einen Beschleunigungsvorgang (+1) umgewandelt wird.
	
	Eine Strecke lässt sich befahren, wenn folgende Bedingungen gegeben sind:
	\begin{enumerate}
		\item Man erreicht mit der maximalen Geschwindigkeit das Ziel.
		\item Man erreicht mit der minimal nötigen Geschwindigkeit das Ziel mit der Geschwindigkeit 0.
	\end{enumerate}

	Während dieses ersten Teils – dem Ablaufen der Strecke – muss sich der Fahrradfahrer neben den Geschwindigkeiten zusätzlich noch
	eine weitere Größe merken: Die Anzahl der Umwandlungen von Brems- in Beschleunigungsvorgänge. Die Anzahl dieser Umwandlungen entspricht
	den zu Beginn nötigen Beschleunigungen auf geraden Streckenabschnitten. Auf den restlichen geraden Streckenabschnitten muss abgebremst werden.
\subsection{Laufzeit}
	Die Laufzeit ist allein abhängig von der Länge n der Strecke, da für jeden Streckenabschnitt die gleichen Operationen ausgeführt werden müssen. Laufzeitkomplexität: 

	\[
		\mathcal O(n)
	\]

	\clearpage
	\section{Umsetzung}
		
	Zur Umsetzung nutzen wir die Klasse Fahrradfahrer.
	Diese Klasse hat die Funktionen ablaufen() und fahren().
	In der Funktion ablaufen() ist dabei die koplette Logik zum Entscheiden, ob eine Strecke befahrbar ist, und zur Berechnung der nötigen Beschleunigungen impementiert. 
	Dabei werden die Anzahl der nötigen Beschleunigungen in der Variable Beschleunigungen gespeichert.
	In fahren() wird das Ergebins aus ablaufen() nochmals überprüft und die Anweisungen für die geraden Streckenabschnitte werden ausgegeben. 
	Theoretisch ist das aufrufen der Funktion fahren() aber redundant: Es werden keine neuen Erkenntnisse gewonnen.
	

	\clearpage
	\section{Beispiel}
		Der Compilierungsbefehl lautet:

\shellcmd{g++ fahrradfahrer.cpp fahrradfahrer.h main.cpp -o fahrspass.out}

Von den Parcours ließen sich 5 und 9 befahren. Die Anderen waren nicht befahrbar. Von dieser Aussage sind die Parcours A, B und C ausgenommen.
Das Programm benötigt für den Parcour 9 ca. 21 Sekunden. Für die anderen Beispiele entsprechend weniger (Laufzeitkomplexität: \(O(n)\)).
Die Ausgabe im gewünschten Format kann in seperaten Dateien eingesehen werden.
\subsection{Parcour 4}
	Programmausgabe:\\
	\shellout{  \$ ./fahrspass.out \\
				Datei : parcours4.txt\\
				Die Strecke lässt sich befahren. Man muss am Anfang 1001577 mal beschleunigen.\\
				Endgeschwindigkeit : 0
}
\subsection{Parcour 6}
	Programmausgabe:\\
	\shellout{  \$ ./fahrspass.out \\
				Datei : parcours6.txt \\
				Die Strecke lässt sich nicht befahren.
}
\subsection{Parcour 9}
	Programmausgabe:\\
	\shellout{ \$ ./fahrspass.out \\
				Datei : parcours9.txt \\
				Die Strecke lässt sich befahren. Man muss am Anfang 49992957 mal beschleunigen. \\
				Endgeschwindigkeit : 0 \\
}


	\clearpage
	\section{Quellcode}
		\subsection{Ablaufen}
		\lstinputlisting[firstline=18,lastline=67]{code/fahrradfahrer.cpp}
\subsection{Fahren}
		\lstinputlisting[firstline=69,lastline=106]{code/fahrradfahrer.cpp}
\subsection{Abbruch}
		\lstinputlisting[firstline=107,lastline=111]{code/fahrradfahrer.cpp}

\end{document}
