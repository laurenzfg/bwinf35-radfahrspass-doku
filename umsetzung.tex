
	Zur Umsetzung nutzen wir die Klasse Fahrradfahrer.
	Diese Klasse hat die Funktionen ablaufen() und fahren().
	In der Funktion ablaufen() ist dabei die koplette Logik zum Entscheiden, ob eine Strecke befahrbar ist, und zur Berechnung der nötigen Beschleunigungen impementiert. 
	Dabei werden die Anzahl der nötigen Beschleunigungen in der Variable Beschleunigungen gespeichert.
	In fahren() wird das Ergebins aus ablaufen() nochmals überprüft und die Anweisungen für die geraden Streckenabschnitte werden ausgegeben. 
	Theoretisch ist das aufrufen der Funktion fahren() aber redundant: Es werden keine neuen Erkenntnisse gewonnen.
	
