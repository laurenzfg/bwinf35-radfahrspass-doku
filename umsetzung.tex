	Zur Umsetzung nutzen wir die Klasse Fahrradfahrer.
	
	Diese Klasse hat die Funktionen \texttt{ablaufen()} und \texttt{fahren()}.
	Die Funktion \texttt{ablaufen()} spiegelt den "`Icandothat-Poker"' wieder, bei dem entschieden wird, ob eine Strecke befahrbar ist. Die Anzahl der nötigen Beschleunigungen wird wie in der Lösungsidee dargestellt berechnet. Diese Anzahl wird in der Variablen \texttt{Beschleunigungen} gespeichert.

	Die Funktion \texttt{fahren()} entspricht dem eigentlichen Abfahren: Hier wird das Ergebnis aus \texttt{ablaufen()} getestet. Außerdem werden die Anweisungen für die geraden Streckenabschnitte im gewünschten Ausgabeformat in die Datei "`prot.txt"' geschrieben. 

	Allerdings ist das Aufrufen der Funktion \texttt{fahren()} für die Aufgabenbearbeitung nicht zwingend erforderlich: Es werden keine neuen Erkenntnisse gewonnen, die Anzahl der Anfangs nötigen Beschleunigungen ist vorher bekannt.  Ob und wie eine Strecke befahren werden kann, stellt \texttt{ablaufen()} allein fest. Somit dient \texttt{fahren()} nur der Überprüfung und der Ausgabe.