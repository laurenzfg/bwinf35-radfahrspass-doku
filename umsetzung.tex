
	Zur Umsetzung nutzen wir die Klasse Fahrradfahrer.
	Diese Klasse hat die Funktionen \texttt{ablaufen()} und \texttt{fahren()}.
	In der Funktion \texttt{ablaufen()} ist dabei die komplette Logik zum Entscheiden, ob eine Strecke befahrbar ist und die 
	zur Berechnung der nötigen Beschleunigungen, implementiert. 
	Dabei werden die Anzahl der nötigen Beschleunigungen in der Variable \texttt{Beschleunigungen} gespeichert.
	In \texttt{fahren()} wird das Ergebnis aus \texttt{ablaufen()} nochmals überprüft und die Anweisungen für die geraden Streckenabschnitte werden ausgegeben. 
	Theoretisch ist das Aufrufen der Funktion \texttt{fahren()} aber redundant: Es werden keine neuen Erkenntnisse gewonnen. 
	Ob und wie eine Strecke befahren werden kann, stellt \texttt{ablaufen()} allein fest.
	
