\subsection {Lösungsidee}
	Die Strecke wird mit einer theoretisch minimalen Geschwindigkeit und einer theoretisch maximalen Geschwindigkeit befahren. 
	Für die maximale Geschwindigkeit wird auf jedem geraden Streckenabschnitt beschleunigt, für die minimale Geschwindigkeit zunächst auf jedem geraden Streckenabschnitt gebremst.

	Unterschreitet die minimale Geschwindigkeit jedoch 0, wird diese um 2 erhöht, sodass dieser Punkt mit einer positiven Geschwindigkeit erreicht wird.
	Diese Erhöhung entspricht der nachträglichen Umwandlung eines vergangenen Abbremsvorganges in einen Beschleunigungsvorgang. Die Geschwindigkeit wird um zwei erhöht, da ein Bremsvorgang (-1) in einen Beschleunigungsvorgang (+1) umgewandelt wird.
	
	Eine Strecke lässt sich befahren, wenn folgende Bedingungen gegeben sind:
	\begin{enumerate}
		\item Man erreicht mit der maximalen Geschwindigkeit das Ziel.
		\item Man erreicht mit der minimal nötigen Geschwindigkeit das Ziel mit der Geschwindigkeit 0.
	\end{enumerate}

	Während dieses ersten Teils – dem Ablaufen der Strecke – muss sich der Fahrradfahrer neben den Geschwindigkeiten zusätzlich noch
	eine weitere Größe merken: Die Anzahl der Umwandlungen von Brems- in Beschleunigungsvorgänge. Die Anzahl dieser Umwandlungen entspricht
	den zu Beginn nötigen Beschleunigungen auf geraden Streckenabschnitten. Auf den restlichen geraden Streckenabschnitten muss abgebremst werden.
\subsection{Laufzeit}
	Die Laufzeit ist allein abhängig von der Länge n der Strecke, da für jeden Streckenabschnitt die gleichen Operationen ausgeführt werden müssen. Laufzeitkomplexität: \(O(n)\)
