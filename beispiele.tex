Der Kompilierungsbefehl lautet:

\shellcmd{g++ fahrradfahrer.cpp fahrradfahrer.h main.cpp -o fahrspass.out}

Es lassen sich Parcours C, 4 und 9 befahren. Die anderen Parcours sind nicht befahrbar.
Das Programm benötigt für den Parcour 9 ungefähr 21 Sekunden. Die anderen Beispiele werden deutlich schneller berechnet. (Laufzeitkomplexität: \(\mathcal O(n)\)).

Die Ausgabe im gewünschten Format kann in separaten Dateien eingesehen werden, hier nur eine verkürzte Form:
\subsection{Parcours C}
	Programmausgabe:\\
	\shellout{  \$ ./fahrspass.out \\
				Datei : parcoursC_03.txt\\
				Die Strecke lässt sich befahren. Man muss am Anfang 1 mal beschleunigen.\\
				Endgeschwindigkeit : 0
}	
\subsection{Parcours 4}
	Programmausgabe:\\
	\shellout{  \$ ./fahrspass.out \\
				Datei : parcours4.txt\\
				Die Strecke lässt sich befahren. Man muss am Anfang 1001577 mal beschleunigen.\\
				Endgeschwindigkeit : 0
}
\subsection{Parcours 6}
	Programmausgabe:\\
	\shellout{  \$ ./fahrspass.out \\
				Datei : parcours6.txt \\
				Die Strecke lässt sich nicht befahren.
}
\subsection{Parcours 9}
	Programmausgabe:\\
	\shellout{ \$ ./fahrspass.out \\
				Datei : parcours9.txt \\
				Die Strecke lässt sich befahren. Man muss am Anfang 49992957 mal beschleunigen. \\
				Endgeschwindigkeit : 0 \\
}

