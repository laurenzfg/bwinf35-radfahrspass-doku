Der Compilierungsbefehl lautet:

\shellcmd{g++ fahrradfahrer.cpp fahrradfahrer.h main.cpp -o fahrspass.out}

Von den Parcours ließen sich 5 und 9 befahren. Die Anderen waren nicht befahrbar. Von dieser Aussage sind die Parcours A, B und C ausgenommen.
Das Programm benötigt für den Parcour 9 ca. 21 Sekunden. Für die anderen Beispiele entsprechend weniger (Laufzeitkomplexität: \(O(n)\)).
\subsection{Parcour 4}
	Programmausgabe:\\
	\shellout{  \$ ./fahrspass.out \\
				Datei : parcours4.txt\\
				Die Strecke lässt sich befahren. Man muss am Anfang 1001577 mal Beschleunigungen.\\
				Endgeschwindigkeit : 0
}
\subsection{Parcour 6}
	Programmausgabe:\\
	\shellout{  \$ ./fahrspass.out \\
				Datei : parcours6.txt \\
				Die Strecke lässt sich nicht befahren.
}
\subsection{Parcour 9}
	Programmausgabe:\\
	\shellout{ \$ ./fahrspass.out \\
				Datei : parcours9.txt \\
				Die Strecke lässt sich befahren. Man muss am Anfang 49992957 mal Beschleunigungen. \\
				Endgeschwindigkeit : 0 \\
}

